\documentclass[12pt]{article}
\usepackage[utf8]{inputenc}
\usepackage[T1]{fontenc}
\usepackage[polish]{babel}
\usepackage{geometry}
\usepackage{tabularx}
\usepackage[table,xcdraw,dvipsnames]{xcolor}
\usepackage{color}
\usepackage{subfig}
\usepackage{sidecap}
\usepackage{wrapfig}
\usepackage{float}
\usepackage{enumerate}
\usepackage{graphicx}
\usepackage{multirow}
\setlength{\parindent}{0pt}
\usepackage{hyperref}
\usepackage{titlesec}
\titlelabel{\thetitle.\quad}
\usepackage{amsmath}
\usepackage{anyfontsize}
\usepackage{indentfirst}
\usepackage{listings}
\usepackage{multicol}
\usepackage{pgfplots}
\newgeometry{tmargin=1.8cm,bmargin=1.8cm,lmargin =1.8cm,rmargin=1.8cm}

\begin{document}
    
\begin{titlepage}
    \begin{figure}
        \centering
        \includegraphics[width=18cm]{logo-PWr.png}
        
        \label{fig:pwr}
    \end{figure}
        \begin{center}
            \huge Wydział Elektroniki, Fotoniki i Mikrosystemów \\ 
            \vspace{40pt}
            \huge Projektowanie Algorytmów i Metody Sztucznej Inteligencji  \\
        \end{center}
        \vspace{60pt}
        \hrule
        \vspace{1pt}
        \hrule
        \begin{center}
            {\fontsize{40}{50}\selectfont Kółko i krzyżyk\\ }
            \vspace{10pt}
            {\fontsize{25}{25}\selectfont Projekt 3 - zadanie na ocenę bdb  }
        \end{center}
        \hrule
        \vspace{1pt}
        \hrule
        \begin{flushright}
            \vspace{65pt}
            \textit{\Large Prowadzący:}\\
            
            \Large Dr inż. Łukasz Jeleń\\
            \vspace{10pt}
            \textit{\Large Wykonała:}\\
            \Large Zuzanna Mejer, 259382 \\
        
        \end{flushright}
        \vspace{65pt}
        \begin{center}
            \large Wrocław, \today r.
        \end{center}
    \end{titlepage}

\section{Wstępne założenia}
Zadanie polegało na zaprojektowaniu i zaimplementowaniu algorytmu, który radziłby sobie z sortowaniem otrzymanych 
pakietów składających się z numeru i tekstu wiadomości. Do wykonania zadania przyjęto pewne założenia. 
Pierwszym z nich jest przyjęcie perspektywy Anny i jej komputera. Oznacza to tyle, że napisany program nie zajmuje się 
dzieleniem wiadomości na \textit{n} pakietów oraz nie nadaje im numerów, gdyż wymienione czynności wykonywał,
zgodnie z poleceniem, Jan na swoim komputerze. Zatem program został przygotowany tak, żeby pracować już z 
losowo ułożonymi i ponumerowanymi pakietami. Drugim przyjętym założeniem jest przesyłanie tych pakietów w jednym 
pliku tekstowym jako drugi argument wywołania podczas uruchamiania pliku wykonywalnego. Kolejnym założeniem jest
forma zakończenia programu i wyświetlenie posortowanej wiadomości. Zdecydowano, że posortowana wiadomość
nie tylko będzie wyświetlana na standardowym wyjściu, ale również będzie zapisywana do nowo 
tworzonego pliku nazwanego ,,uporządkowany\_list.txt''. 

\section{Struktura danych}
W celu zrealizowania zadania wybrano strukturę danych do przechowywania informacji (numeru i tekstu wiadomości), 
jaką jest kolejka priorytetowa na bazie listy dwukierunkowej. Poniżej przedstawiono uzasadnienie wyboru.
Kolejka priorytetowa jest abstrakcyjnym typem danych, służącym do przechowywania 
zbioru elementów, przy czym każdy element posiada dodatkowe pole do przechowywania priorytetu lub inaczej 
klucza. Zatem, z założenia, do kolejki priorytetowej można wprowadzać takie pakiety, składające się 
z numeru (priorytetu) i tekstu. Kolejnym argumentem przemawiającym za użyciem kolejki priorytetowej 
jest układanie elementów w kolejce nie w kolejności wprowadzania, a w kolejności priorytetu rosnąco lub malejąco.
Jako implementację kolejki priorytetowej wybrano listę dwukierunkową. Umożliwia ona wstawianie elementu 
w dowolnym miejscu, nie tylko na początku lub na końcu listy, co pozwala na wydajne sortowanie elementów przy ich 
wstawianiu do kolejki, a więc tworzenie listy uporządkowanej. Analogicznie, lista dwukierunkowa pozwala także na 
usuwanie elementów z dowolnego miejsca.

Operacje na kolejce priorytetowej uwzględniają takie metody jak: 
\begin{itemize}
    \item umieszczanie nowego elementu w kolejce,
    \item usuwanie elementu z kolejki,
    \item zwracanie informacji o tym, czy kolejka jest pusta czy nie,
    \item informowanie o liczbie elementów umieszczonych w kolejce,
    \item zwracanie wartości pierwszego elementu w kolejce.
\end{itemize}
Zaimplementowanie powyższych metod będzie omówione poniżej.


\section{Omówienie programu}
\subsection{Obsługa plików}
Do obsługi plików została stworzona klasa o nazwie \textit{file} z dwoma prywatnymi polami 
określającymi plik wejściowy \textit{in\_file} oraz plik wyjściowy \textit{out\_file}.
Stworzone zostały metody służące do otwierania i zamykania plików wejściowych i wyjściowych 
(\textit{open\_in\_file, open\_out\_file, close\_in\_file, close\_out\_file}), 
czytania (\textit{read\_file}) i sprawdzania końca (\textit{end\_of\_file}) pliku wejściowego
oraz zapisywania do pliku wyjściowego (\textit{write\_out\_file}).

\colorbox{Dandelion}{+ listing kodu} 

Plik wejściowy podawany przy uruchomieniu programu ma domyślnie formę przedstawioną 
na poniższym zdjęciu. Na początku linii jest numer pakietu, a po spacji wiadomość. 

\colorbox{Dandelion}{+ zdjęcie formy pliku}


\subsection{Kolejka priorytetowa jako lista dwukierunkowa}
Do zaimplementowania kolejki na liście dwukierunkowej, została stworzona struktura \textit{node}, 
która ma dwa pola do przechowywania pakietu: \textit{key} oraz \textit{text}, oraz dwa wskaźniki 
na następny i poprzedni element: \textit{*next} i \textit{*prev}. Została stworzona klasa 
\textit{priority\_queue} z pierwszym wskaźnikiem na początek kolejki: \textit{*header} 
typu struktury \textit{node}. Zostały zaimplementowane metody:
\begin{itemize}
    \item \textit{comparison} - służy do porównywania wartości dwóch kluczy i jest 
    wykorzystywana przy sortowaniu elementów podczas wprowadzania do kolejki.
    \item \textit{insert} - odpowiednik metody \textit{push}; służy do wstawiania elementów 
    do kolejki. Metoda \textit{insert} różni się tym, że umożliwia umieszczenie elementu w 
    dowolnym miejscu w kolejce, tak, aby tworzyć kolejkę uporządkowaną rosnąco. Przykładowo, jeżeli
    do metody \textit{insert} zostanie przekazany pakiet z pliku o wartości klucza 4, a w kolejce
    będą już elementy o wartościach kluczy 1,2,6,8, to metoda ta, wykorzystując metodę \textit{comparison}
    znajdzie pierwszą większą wartość klucza od pobranej (4) i wstawi ją przed ten element (6).
    \item \textit{display\_text} - odpowiednik metody \textit{top}; odpowiada za
    wyświetlanie samego tekstu pierwszego elementu w kolejce.
    \item \textit{remove\_minimum} - odpowiednik \textit{pop}; służy do usuwania elementu z 
    początku kolejki. 
    \item \textit{empty} - zwraca informację o tym, czy kolejka jest pusta.
    \item \textit{return\_minimum} - jest funkcją pomocniczą do zwracania tekstu pierwszego 
    elementu w kolejce. Wykorzystuje ją metoda zapisywania do pliku. 
    \item \textit{size} - zwraca ilość elementów umieszczonych w kolejce priorytetowej.
\end{itemize}

\colorbox{Dandelion}{+listing kodu}

\subsection{Działanie programu}
Program wykonuje następujące działania:
\begin{enumerate}
    \item Otwiera zarówno plik wejściowy, podany jako argument wywołania pliku wykonywalnego, 
    jak i wyjściowy, do którego będą zapisywane dane.
    \item Aż do natrafienia na koniec danych w pliku, program sczytuje kolejne sekwencje, 
    tworzy nowy element typu \textit{node}, przypisuje do niego wartości klucza i wiadomości,
    a następnie wstawia w odpowiednie miejsce w kolejce, sortując rosnąco względem wartości klucza.  
    \item Kiedy nie ma więcej danych w pliku i kolejka jest już utworzona, następuje wyświetlenie
    tekstu elementu o najmniejszej wartości klucza na standardowe wyjście, zapisanie go do pliku 
    wyjściowego oraz usunięcie go z kolejki priorytetowej. Ten etap powtarza się aż do wyczyszczenia 
    całej kolejki.   
\end{enumerate}


\colorbox{Dandelion}{+listing kodu}

\section{Analiza złożoności obliczeniowej}


\section{Testy}


\section{Podsumowanie i wnioski}


\section{Literatura}
\begin{enumerate}
    \footnotesize
    \item Wykład ,,Projektowanie Algorytmów i Metody Sztucznej Inteligencji''
    \item M.T.Goodrich, R. Tamassia, D.Mount ,,Data Structures \& Algorithms in C++''
    \item \url{ https://cpp0x.pl/kursy/Kurs-STL-C++/Adapter-kolejki-priorytetowej-std-priority_queue/118} (dostęp: 31.03.2022)
\end{enumerate}

\end{document}