\subsection{Przebieg eksperymentów}
Sortowane były jedynie oceny filmów. Sortowania odbyły się dla 10 000, 100 000, 500 000 oraz maksymalnej ilości danych z pliku po przefiltrowaniu. Dla każdego zestawu danych wykonano po 100 pomiarów.



\subsection{Sortowanie przez scalanie}

\begin{figure}[H]
    \centering
    \subfloat[10 000 elementów]
    {
        \begin{tikzpicture}[scale=0.9]
            \begin{axis}
                [
                    /pgf/number format/.cd,use comma,1000 sep={},   %europejski format
                    ylabel={czas},
                    xlabel={liczba elementów},
                    xmin=0, xmax=10000,
                    ymin=0, ymax=1295207,
                    ymajorgrids=true, xmajorgrids=true, grid style=dashed
                ]
            \addplot[color=WildStrawberry, only marks, mark size=0.04cm] table {przefiltrowane_dane_10tys.txt};
            \end{axis}
        \end{tikzpicture}
    } 
    \quad
    \subfloat[100 000 elementów]
    {
        \begin{tikzpicture}[scale=0.9]
            \begin{axis}
                [
                    /pgf/number format/.cd,use comma,1000 sep={},   %europejski format
                    ylabel={czas},
                    xlabel={liczba elementów},
                    xmin=0, xmax=100000,
                    ymin=0, ymax=13966928,
                    ymajorgrids=true, xmajorgrids=true, grid style=dashed
                ]
            \addplot[color=WildStrawberry, only marks, mark size=0.04cm] table {przefiltrowane_dane_100tys.txt};
            \end{axis}
        \end{tikzpicture}
    } \quad
    \subfloat[500 000 elementów]
    {
        \begin{tikzpicture}[scale=0.9]
            \begin{axis}
                [
                    /pgf/number format/.cd,use comma,1000 sep={},   %europejski format
                    ylabel={czas},
                    xlabel={liczba elementów},
                    xmin=0, xmax=500000,
                    ymin=0, ymax=70052081,
                    ymajorgrids=true, xmajorgrids=true, grid style=dashed
                ]
            \addplot[color=WildStrawberry, only marks, mark size=0.04cm] table {przefiltrowane_dane_500tys.txt};
            \end{axis}
        \end{tikzpicture}
    } \quad
    \subfloat[maksymalna liczba elementów]
    {
        \begin{tikzpicture}[scale=0.9]
            \begin{axis}
                [
                    /pgf/number format/.cd,use comma,1000 sep={},   %europejski format
                    ylabel={czas},
                    xlabel={liczba elementów},
                    xmin=0, xmax=1000000,
                    ymin=0, ymax=138481693,
                    ymajorgrids=true, xmajorgrids=true, grid style=dashed
                ]
            \addplot[color=WildStrawberry, only marks, mark size=0.04cm] table {przefiltrowane_dane_max.txt};
            \end{axis}
        \end{tikzpicture}
    }
    \caption{Sortowanie przez scalanie dla różnej liczby elementów}
    \label{fig: scalanie}
\end{figure}



\begin{table}[H]
    \centering
    \caption{Dokładny i przybliżony czas sortowania przez scalanie}
    \label{tab: czas_scalanie}
    \begin{tabular}{c|cccc|}
    \cline{2-5}
                                                             & \multicolumn{4}{c|}{liczba elementów}                                                                     \\ \cline{2-5} 
                                                             & \multicolumn{1}{c|}{10 000}  & \multicolumn{1}{c|}{100 000}  & \multicolumn{1}{c|}{500 000}  & maksymalna \\ \hline
    \multicolumn{1}{|c|}{dokładny czas sortowań {[}ns{]}}    & \multicolumn{1}{c|}{1295207} & \multicolumn{1}{c|}{13966928} & \multicolumn{1}{c|}{70052081} & 138481693  \\ \hline
    \multicolumn{1}{|c|}{przybliżony czas sortowań {[}ms{]}} & \multicolumn{1}{c|}{1,29}    & \multicolumn{1}{c|}{13,97}    & \multicolumn{1}{c|}{70,05}    & 138,48     \\ \hline
    \end{tabular}
    \end{table}


Algorytm sortowania przez scalanie wykazał złożoność obliczeniową liniową dla każdego zestawu danych. Czas sortowania dla maksymalnej liczby elementów z pliku wyniósł \textbf{138,48 ms $\approx$ 0,14 s}.





\subsection{Sortowanie szybkie}


\begin{figure}[H]
    \centering
    \subfloat[10 000 elementów]
    {
        \begin{tikzpicture}[scale=0.9]
            \begin{axis}
                [
                    /pgf/number format/.cd,use comma,1000 sep={},   %europejski format
                    ylabel={czas},
                    xlabel={liczba elementów},
                    xmin=0, xmax=10000,
                    ymin=0, ymax=1489031,
                    ymajorgrids=true, xmajorgrids=true, grid style=dashed
                ]
            \addplot[color=WildStrawberry, only marks, mark size=0.04cm] table {quick_przefiltrowane_dane_10tys.txt};
            \end{axis}
        \end{tikzpicture}
    } 
    \quad
    \subfloat[100 000 elementów]
    {
        \begin{tikzpicture}[scale=0.9]
            \begin{axis}
                [
                    /pgf/number format/.cd,use comma,1000 sep={},   %europejski format
                    ylabel={czas},
                    xlabel={liczba elementów},
                    xmin=0, xmax=100000,
                    ymin=0, ymax=11705054,
                    ymajorgrids=true, xmajorgrids=true, grid style=dashed
                ]
            \addplot[color=WildStrawberry, only marks, mark size=0.04cm] table {quick_przefiltrowane_dane_100tys.txt};
            \end{axis}
        \end{tikzpicture}
    } \quad
    \subfloat[500 000 elementów]
    {
        \begin{tikzpicture}[scale=0.9]
            \begin{axis}
                [
                    /pgf/number format/.cd,use comma,1000 sep={},   %europejski format
                    ylabel={czas},
                    xlabel={liczba elementów},
                    xmin=0, xmax=500000,
                    ymin=0, ymax=57671388,
                    ymajorgrids=true, xmajorgrids=true, grid style=dashed
                ]
            \addplot[color=WildStrawberry, only marks, mark size=0.04cm] table {quick_przefiltrowane_dane_500tys.txt};
            \end{axis}
        \end{tikzpicture}
    } \quad
    \subfloat[maksymalna liczba elementów]
    {
        \begin{tikzpicture}[scale=0.9]
            \begin{axis}
                [
                    /pgf/number format/.cd,use comma,1000 sep={},   %europejski format
                    ylabel={czas},
                    xlabel={liczba elementów},
                    xmin=0, xmax=1000000,
                    ymin=0, ymax=106918054,
                    ymajorgrids=true, xmajorgrids=true, grid style=dashed
                ]
            \addplot[color=WildStrawberry, only marks, mark size=0.04cm] table {quick_przefiltrowane_dane_max.txt};
            \end{axis}
        \end{tikzpicture}
    }
    \caption{Sortowanie przez scalanie dla różnej liczby elementów}
    \label{fig: scalanie}
\end{figure}


\begin{table}[H]
    \caption{Dokładny i przybliżony czas sortowania szybkiego}
    \label{tab: czas_szybkie}
    \begin{tabular}{c|cccc|}
    \cline{2-5}
                                                             & \multicolumn{4}{c|}{liczba elementów}                                                                     \\ \cline{2-5} 
                                                             & \multicolumn{1}{c|}{10 000}  & \multicolumn{1}{c|}{100 000}  & \multicolumn{1}{c|}{500 000}  & maksymalna \\ \hline
    \multicolumn{1}{|c|}{dokładny czas sortowań {[}ns{]}}    & \multicolumn{1}{c|}{1489031} & \multicolumn{1}{c|}{11705054} & \multicolumn{1}{c|}{57671388} & 106918054  \\ \hline
    \multicolumn{1}{|c|}{przybliżony czas sortowań {[}ms{]}} & \multicolumn{1}{c|}{1,49}    & \multicolumn{1}{c|}{11,71}    & \multicolumn{1}{c|}{57,67}    & 106,92     \\ \hline
    \end{tabular}
    \end{table}



Dla małej ilości danych (10 000), algorytm sortowania szybkiego wykazał cechy złożoności kwadratowej, natomiast dla większych ilości danych - liniowej. Czas sortowania dla maksymalnej liczby elementów z pliku wyniósł \textbf{106,92 ms $\approx$ 0,11 s}.




\subsection{Sortowanie introspektywne}

\begin{figure}[H]
    \centering
    \subfloat[10 000 elementów]
    {
        \begin{tikzpicture}[scale=0.9]
            \begin{axis}
                [
                    /pgf/number format/.cd,use comma,1000 sep={},   %europejski format
                    ylabel={czas},
                    xlabel={liczba elementów},
                    xmin=0, xmax=10000,
                    ymin=0, ymax=2873127,
                    ymajorgrids=true, xmajorgrids=true, grid style=dashed
                ]
            \addplot[color=WildStrawberry, only marks, mark size=0.04cm] table {intro_przefiltrowane_dane_10tys.txt};
            \end{axis}
        \end{tikzpicture}
    } 
    \quad
    \subfloat[100 000 elementów]
    {
        \begin{tikzpicture}[scale=0.9]
            \begin{axis}
                [
                    /pgf/number format/.cd,use comma,1000 sep={},   %europejski format
                    ylabel={czas},
                    xlabel={liczba elementów},
                    xmin=0, xmax=100000,
                    ymin=0, ymax=36543660,
                    ymajorgrids=true, xmajorgrids=true, grid style=dashed
                ]
            \addplot[color=WildStrawberry, only marks, mark size=0.04cm] table {intro_przefiltrowane_dane_100tys.txt};
            \end{axis}
        \end{tikzpicture}
    } \quad
    \subfloat[500 000 elementów]
    {
        \begin{tikzpicture}[scale=0.9]
            \begin{axis}
                [
                    /pgf/number format/.cd,use comma,1000 sep={},   %europejski format
                    ylabel={czas},
                    xlabel={liczba elementów},
                    xmin=0, xmax=500000,
                    ymin=0, ymax=148902464,
                    ymajorgrids=true, xmajorgrids=true, grid style=dashed
                ]
            \addplot[color=WildStrawberry, only marks, mark size=0.04cm] table {intro_przefiltrowane_dane_500tys.txt};
            \end{axis}
        \end{tikzpicture}
    } \quad
    \subfloat[maksymalna liczba elementów]
    {
        \begin{tikzpicture}[scale=0.9]
            \begin{axis}
                [
                    /pgf/number format/.cd,use comma,1000 sep={},   %europejski format
                    ylabel={czas},
                    xlabel={liczba elementów},
                    xmin=0, xmax=1000000,
                    ymin=0, ymax=292090659,
                    ymajorgrids=true, xmajorgrids=true, grid style=dashed
                ]
            \addplot[color=WildStrawberry, only marks, mark size=0.04cm] table {intro_przefiltrowane_dane_max.txt};
            \end{axis}
        \end{tikzpicture}
    }
    \caption{Sortowanie introspektywne dla różnej liczby elementów}
    \label{fig: introspektywne}
\end{figure}

\begin{table}[H]
    \caption{Dokładny i przybliżony czas sortowania introspektywnego}
    \label{tab: czas_introspektywne}
    \begin{tabular}{c|cccc|}
    \cline{2-5}
                                                             & \multicolumn{4}{c|}{liczba elementów}                                                                     \\ \cline{2-5} 
                                                             & \multicolumn{1}{c|}{10 000} & \multicolumn{1}{c|}{100 000}  & \multicolumn{1}{c|}{500 000}   & maksymalna \\ \hline
    \multicolumn{1}{|c|}{dokładny czas sortowań {[}ns{]}}    & \multicolumn{1}{c|}{287312} & \multicolumn{1}{c|}{36543660} & \multicolumn{1}{c|}{148902464} & 292090659  \\ \hline
    \multicolumn{1}{|c|}{przybliżony czas sortowań {[}ms{]}} & \multicolumn{1}{c|}{0,29}   & \multicolumn{1}{c|}{36,54}    & \multicolumn{1}{c|}{148,90}    & 292,09     \\ \hline
    \end{tabular}
\end{table}

Algorytm sortowania introspektywnego wykazał złożoność obliczeniową liniową dla każdego zestawu
danych. Czas sortowania dla maksymalnej liczby elementów z pliku wyniósł \textbf{292,09 ms $\approx$ 0,29 s}.
