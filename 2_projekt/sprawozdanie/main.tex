\documentclass[12pt]{article}
\usepackage[utf8]{inputenc}
\usepackage[T1]{fontenc}
\usepackage[polish]{babel}
\usepackage{geometry}
\usepackage{tabularx}
\usepackage[table,xcdraw,dvipsnames]{xcolor}
\usepackage{color}
\usepackage{subfig}
\usepackage{sidecap}
\usepackage{wrapfig}
\usepackage{float}
\usepackage{enumerate}
\usepackage{graphicx}
\usepackage{multirow}
\setlength{\parindent}{0pt}
\usepackage{hyperref}
\usepackage{titlesec}
\titlelabel{\thetitle.\quad}
\usepackage{amsmath}
\usepackage{anyfontsize}
\usepackage{indentfirst}
\usepackage{listings}
\usepackage{multicol}
\usepackage{pgfplots}
\usepackage{listings}
\usepackage{fancyhdr}
\newgeometry{tmargin=1.8cm,bmargin=1.8cm,lmargin =1.8cm,rmargin=1.8cm}


\newgeometry{tmargin=1.8cm,bmargin=1.8cm,lmargin =1.8cm,rmargin=1.8cm}
\pagestyle{fancy}
\fancyhf{}
\rhead{\textit{Zuzanna Mejer} }
\lhead{\textit{Algorytmy sortowania} }
\cfoot{\thepage}


\begin{document}
    
\begin{titlepage}
    \begin{figure}
        \centering
        \includegraphics[width=18cm]{logo-PWr.png}
        
        \label{fig:pwr}
    \end{figure}
        \begin{center}
            \huge Wydział Elektroniki, Fotoniki i Mikrosystemów \\ 
            \vspace{40pt}
            \huge Projektowanie Algorytmów i Metody Sztucznej Inteligencji  \\
        \end{center}
        \vspace{60pt}
        \hrule
        \vspace{1pt}
        \hrule
        \begin{center}
            {\fontsize{40}{50}\selectfont Kółko i krzyżyk\\ }
            \vspace{10pt}
            {\fontsize{25}{25}\selectfont Projekt 3 - zadanie na ocenę bdb  }
        \end{center}
        \hrule
        \vspace{1pt}
        \hrule
        \begin{flushright}
            \vspace{65pt}
            \textit{\Large Prowadzący:}\\
            
            \Large Dr inż. Łukasz Jeleń\\
            \vspace{10pt}
            \textit{\Large Wykonała:}\\
            \Large Zuzanna Mejer, 259382 \\
        
        \end{flushright}
        \vspace{65pt}
        \begin{center}
            \large Wrocław, \today r.
        \end{center}
    \end{titlepage}
\tableofcontents
\newpage

\section{Wprowadzenie}
Zadanie miało na celu zapoznanie się z algorytmami sortowania oraz przeprowadzenie analizy efektywności wybranych i zaimplementowanych sortowań. Z wymienionych algorytmów wybrałam sortowania: przez scalanie, szybkie oraz introspektywne. 

\section{Opis badanych algorytmów i ich złożoność obliczeniowa}
\subsection{Sortowanie przez scalanie}
Jest to rekurencyjny algorytm sortowania danych, stosujący metodę ,,dziel i zwyciężaj''. W algorytmie wyróżnia się trzy podstawowe kroki: podział danych wejściowych na 2 rozłączne podzbiory; rekurencyjnie zastosowanie sortowania dla każdego podzbioru, aż do uzyskania struktur jednoelementowych; scalenie posortowanych podzbiorów w jeden zbiór. Całkowita złożoność obliczeniowa dla sortowania przez scalanie wynosi $O(n \cdot log n)$, w związku z czym zastosowanie tego sortowania okaże się wydajniejsze dla bardzo dużych tablic.


\subsection{Sortowanie szybkie}
Również jest to algorytm sortowania danych stosujący metodę ,,dziel i zwyciężaj'', nie wykorzystuje on jednak dodatkowych podtablic. Istnieje wiele implementacji sortowania szybkiego, jednak generalna idea jest taka, że wybierany jest jeden element w sortowanej strukturze, który nazywany jest piwotem. Może być to element środkowy, pierwszy, ostatni bądź losowy, przy czym należy pamiętać, że w przypadkach, kiedy piwot jest ciągle maksymalny lub minimalny, występuje najgorsza złożoność obliczeniowa $O(n^2)$. Przy optymalnych wyborach piwotu, złożoność wynosi $O(n \cdot log n)$. 


\subsection{Sortowanie introspektywne}
Jest to odmiana sortowania hybrydowego, które opiera się na spostrzeżeniu, że niewydajne jest wywoływanie ogromnej liczby rekurencji dla małych tablic w algorytmie sortowania szybkiego. Głównym założeniem algorytmu sortowania introspektywnego jest zatem wyeliminowanie problemu złożoności $O(n^2)$ występującej w najgorszym przypadku sortowania szybkiego. Sortowanie introspektywne jest połączeniem sortowania szybkiego i sortowania przez kopcowanie, które jest traktowane jako pomocnicze. Tym samym złożoność obliczeniowa wynosi $O(n \cdot log n)$.


\subsection{Porównanie złożoności obliczeniowych wybranych algorytmów}

Poniższa tabela zestawia oczekiwane i najgorsze przypadki złożoności wybranych algorytmów sortowania. Poniżej dodano także poglądowy wykres funkcji, na którym widać, że dla małej liczby danych sortowanie o złożoności kwadratowej będzie wydajniejsze niż dla logarytmicznej i przeciwnie dla dużej liczby elementów do posortowania.

\begin{table}[H]
    \centering
    \renewcommand{\tablename}{Tab.}
    \caption{Porównanie oczekiwanych i najgorszych przypadków złożoności obliczeniowej dla wybranych algorytmów sortowania}
    \label{tab: opis_zlozonosc}
    \begin{tabular}{c|ccc|}
        \cline{2-4}
        & \multicolumn{3}{c|}{sortowanie}   \\ \cline{2-4} & \multicolumn{1}{c|}{przez scalanie} & \multicolumn{1}{c|}{szybkie} & introspektywne \\ \hline
        \multicolumn{1}{|c|}{typowa złożoność} & \multicolumn{1}{c|}{$O(nlong)$} & \multicolumn{1}{c|}{$O(nlogn)$} & $O(nlogn)$       \\ \hline
        \multicolumn{1}{|c|}{najgorszy przypadek złożoności} & \multicolumn{1}{c|}{$O(nlogn)$} & \multicolumn{1}{c|}{$O(n^2)$} & $O(nlogn)$       \\ \hline
    \end{tabular}
\end{table}

\begin{figure}[H]
    \centering
    \begin{tikzpicture}[scale=1.4]
        \begin{axis}
            [
                /pgf/number format/.cd,use comma,1000 sep={},   %europejski format
                % width={\textwidth},
                % height={\textheight},
                % scale only axis=true,
                ylabel={czas},
                xlabel={liczba elementów},
                xmin=0, xmax=10,
                ymin=0, ymax=10,
                ymajorgrids=true, xmajorgrids=true, grid style=dashed
            ]
        \addplot[color=WildStrawberry, samples=1000, line width=1pt, domain=0:30] {x};
        \addplot[color=BurntOrange, samples=1000, line width=1pt, domain=0:30] {x^2};
        \addplot[color=Cerulean, samples=1000, line width=1pt, domain=0:30] {x*log2 x};
        \addplot[color=OliveGreen, samples=1000, line width=1pt, domain=0:30] {log2 x};
        \legend {$f(x)=x$ \\ $f(x) = x^2 $ \\ $f(x)=x*log2 x$ \\ $f(x)=log2 x$ \\};
        \end{axis}
    \end{tikzpicture}
    \renewcommand{\figurename}{Rys.}
    \caption{}
    \label{fig: }
\end{figure}




\section{Zadanie 1 - przeszukanie i przefiltrowanie danych}
\subsection{Krótki opis}
Plik udostępniony do sortowania był okrojoną bazą filmów ,,IMDb Largest Review Dataset'' ze strony kaggle.com. Plik zawierał tytuły filmów oraz przypisane im oceny. Niektóre pola z ocenami były puste, zatem przed wykonaniem zadań związanych z sortowaniem, należało wykonać przeszukanie i usunięcie wpisów bez ocen. Do wykonania tego zadania, zastosowano gotową strukturę z biblioteki STL: \textit{std::vector}. Mimo chęci wykonania sortowań na strukturze dwuelementowej: \textit{std::vector< std::pair <std::string, float> >}, przechowującej i tytuł filmu, i ocenę, komputery, na których wykonywałam testy złożoności obliczeniowej, nie były w stanie wykonać sortowań dla maksymalnej liczby elementów z pliku. Podsumowując, wykonane zostało przeszukiwanie, wykorzystujące strukturę jednoelementową, a następnie sortowane były jedynie oceny filmów.
\subsection{Analiza złożoności}
\subsection{Czas przeszukiwania}

\section{Analiza złożoności algorytmów sortowań}
\subsection{Przebieg eksperymentów}
\subsection{Sortowanie przez scalanie}
\subsection{Sortowanie szybkie}
\subsection{Sortowanie introspektywne}


\section{Podsumowanie i wnioski}

\section{Bibliografia}

\end{document}
