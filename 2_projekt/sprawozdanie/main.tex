\documentclass[12pt]{article}
\usepackage[utf8]{inputenc}
\usepackage[T1]{fontenc}
\usepackage[polish]{babel}
\usepackage{geometry}
\usepackage{tabularx}
\usepackage[table,xcdraw,dvipsnames]{xcolor}
\usepackage{color}
\usepackage{subfig}
\usepackage{sidecap}
\usepackage{wrapfig}
\usepackage{float}
\usepackage{enumerate}
\usepackage{graphicx}
\usepackage{multirow}
\setlength{\parindent}{0pt}
\usepackage{hyperref}
\usepackage{titlesec}
\titlelabel{\thetitle.\quad}
\usepackage{amsmath}
\usepackage{anyfontsize}
\usepackage{indentfirst}
\usepackage{listings}
\usepackage{multicol}
\usepackage{pgfplots}
\usepackage{listings}
\usepackage{fancyhdr}
\usepackage{pgfplotstable}
\newgeometry{tmargin=1.8cm,bmargin=1.8cm,lmargin =1.8cm,rmargin=1.8cm}

\newgeometry{tmargin=1.8cm,bmargin=1.8cm,lmargin =1.8cm,rmargin=1.8cm}
\pagestyle{fancy}
\fancyhf{}
\rhead{\textit{Zuzanna Mejer} }
\lhead{\textit{Algorytmy sortowania} }
\cfoot{\thepage}


\begin{document}
\renewcommand{\figurename}{Rys.}
\renewcommand{\tablename}{Tab.}
    
\begin{titlepage}
    \begin{figure}
        \centering
        \includegraphics[width=18cm]{logo-PWr.png}
        
        \label{fig:pwr}
    \end{figure}
        \begin{center}
            \huge Wydział Elektroniki, Fotoniki i Mikrosystemów \\ 
            \vspace{40pt}
            \huge Projektowanie Algorytmów i Metody Sztucznej Inteligencji  \\
        \end{center}
        \vspace{60pt}
        \hrule
        \vspace{1pt}
        \hrule
        \begin{center}
            {\fontsize{40}{50}\selectfont Kółko i krzyżyk\\ }
            \vspace{10pt}
            {\fontsize{25}{25}\selectfont Projekt 3 - zadanie na ocenę bdb  }
        \end{center}
        \hrule
        \vspace{1pt}
        \hrule
        \begin{flushright}
            \vspace{65pt}
            \textit{\Large Prowadzący:}\\
            
            \Large Dr inż. Łukasz Jeleń\\
            \vspace{10pt}
            \textit{\Large Wykonała:}\\
            \Large Zuzanna Mejer, 259382 \\
        
        \end{flushright}
        \vspace{65pt}
        \begin{center}
            \large Wrocław, \today r.
        \end{center}
    \end{titlepage}
\tableofcontents
\newpage

\section{Wprowadzenie}
Zadanie miało na celu zapoznanie się z algorytmami sortowania oraz przeprowadzenie analizy efektywności wybranych i zaimplementowanych sortowań. Z wymienionych algorytmów wybrałam sortowania: przez scalanie, szybkie oraz introspektywne. 

\section{Opis badanych algorytmów i ich złożoność obliczeniowa}
\subsection{Sortowanie przez scalanie}
Jest to rekurencyjny algorytm sortowania danych, stosujący metodę ,,dziel i zwyciężaj''. W algorytmie wyróżnia się trzy podstawowe kroki: podział danych wejściowych na 2 rozłączne podzbiory; rekurencyjnie zastosowanie sortowania dla każdego podzbioru, aż do uzyskania struktur jednoelementowych; scalenie posortowanych podzbiorów w jeden zbiór. Całkowita złożoność obliczeniowa dla sortowania przez scalanie wynosi $O(n \cdot log n)$, w związku z czym zastosowanie tego sortowania okaże się wydajniejsze dla bardzo dużych tablic.


\subsection{Sortowanie szybkie}
Również jest to algorytm sortowania danych stosujący metodę ,,dziel i zwyciężaj'', nie wykorzystuje on jednak dodatkowych podtablic. Istnieje wiele implementacji sortowania szybkiego, jednak generalna idea jest taka, że wybierany jest jeden element w sortowanej strukturze, który nazywany jest piwotem. Może być to element środkowy, pierwszy, ostatni bądź losowy, przy czym należy pamiętać, że w przypadkach, kiedy piwot jest ciągle maksymalny lub minimalny, występuje najgorsza złożoność obliczeniowa $O(n^2)$. Przy optymalnych wyborach piwotu, złożoność wynosi $O(n \cdot log n)$. 


\subsection{Sortowanie introspektywne}
Jest to odmiana sortowania hybrydowego, które opiera się na spostrzeżeniu, że niewydajne jest wywoływanie ogromnej liczby rekurencji dla małych tablic w algorytmie sortowania szybkiego. Głównym założeniem algorytmu sortowania introspektywnego jest zatem wyeliminowanie problemu złożoności $O(n^2)$ występującej w najgorszym przypadku sortowania szybkiego. Sortowanie introspektywne jest połączeniem sortowania szybkiego i sortowania przez kopcowanie, które jest traktowane jako pomocnicze. Tym samym złożoność obliczeniowa wynosi $O(n \cdot log n)$.


\subsection{Porównanie złożoności obliczeniowych wybranych algorytmów}

Poniższa tabela zestawia oczekiwane i najgorsze przypadki złożoności wybranych algorytmów sortowania. Poniżej dodano także poglądowy wykres funkcji, na którym widać, że dla małej liczby danych sortowanie o złożoności kwadratowej będzie wydajniejsze niż dla logarytmicznej i przeciwnie dla dużej liczby elementów do posortowania.

\begin{table}[H]
    \centering
    \renewcommand{\tablename}{Tab.}
    \caption{Porównanie oczekiwanych i najgorszych przypadków złożoności obliczeniowej dla wybranych algorytmów sortowania}
    \label{tab: opis_zlozonosc}
    \begin{tabular}{c|ccc|}
        \cline{2-4}
        & \multicolumn{3}{c|}{sortowanie}   \\ \cline{2-4} & \multicolumn{1}{c|}{przez scalanie} & \multicolumn{1}{c|}{szybkie} & introspektywne \\ \hline
        \multicolumn{1}{|c|}{typowa złożoność} & \multicolumn{1}{c|}{$O(nlong)$} & \multicolumn{1}{c|}{$O(nlogn)$} & $O(nlogn)$       \\ \hline
        \multicolumn{1}{|c|}{najgorszy przypadek złożoności} & \multicolumn{1}{c|}{$O(nlogn)$} & \multicolumn{1}{c|}{$O(n^2)$} & $O(nlogn)$       \\ \hline
    \end{tabular}
\end{table}

\begin{figure}[H]
    \centering
    \begin{tikzpicture}[scale=1.4]
        \begin{axis}
            [
                /pgf/number format/.cd,use comma,1000 sep={},   %europejski format
                % width={\textwidth},
                % height={\textheight},
                % scale only axis=true,
                ylabel={czas},
                xlabel={liczba elementów},
                xmin=0, xmax=10,
                ymin=0, ymax=10,
                ymajorgrids=true, xmajorgrids=true, grid style=dashed
            ]
        \addplot[color=WildStrawberry, samples=1000, line width=1pt, domain=0:30] {x};
        \addplot[color=BurntOrange, samples=1000, line width=1pt, domain=0:30] {x^2};
        \addplot[color=Cerulean, samples=1000, line width=1pt, domain=0:30] {x*log2 x};
        \addplot[color=OliveGreen, samples=1000, line width=1pt, domain=0:30] {log2 x};
        \legend {$f(x)=x$ \\ $f(x) = x^2 $ \\ $f(x)=x*log2 x$ \\ $f(x)=log2 x$ \\};
        \end{axis}
    \end{tikzpicture}
    \renewcommand{\figurename}{Rys.}
    \caption{}
    \label{fig: }
\end{figure}




\section{Implementacja algorytmów sortowania}
\subsection{Sortowanie przez scalanie}


\subsection{Sortowanie szybkie}


\subsection{Sortowanie introspektywne}



\section{Zadanie 1 - przeszukanie i przefiltrowanie danych}
\subsection{Krótki opis}
Plik udostępniony do sortowania był okrojoną bazą filmów ,,IMDb Largest Review Dataset'' ze strony kaggle.com. Plik zawierał tytuły filmów oraz przypisane im oceny. Niektóre pola z ocenami były puste, zatem przed wykonaniem zadań związanych z sortowaniem, należało wykonać przeszukanie i usunięcie wpisów bez ocen. Do wykonania tego zadania, zastosowano gotową strukturę z biblioteki STL: \textit{std::vector}. Mimo chęci wykonania sortowań na strukturze dwuelementowej: \textit{std::vector< std::pair <std::string, float> >}, przechowującej i tytuł filmu, i ocenę, komputery, na których wykonywałam testy złożoności obliczeniowej, nie były w stanie wykonać sortowań dla maksymalnej liczby elementów z pliku. Podsumowując, wykonane zostało przeszukiwanie, wykorzystujące strukturę dwuelementową, lecz sortowane były jedynie oceny filmów. Poniżej przedstawiono algorytm przeszukiwania struktury i usuwania pól z pustymi ocenami:

\lstset{language=C++, firstnumber=1, keywordstyle=\color{blue}, numbers=left, frame = single}
\begin{lstlisting}
    for (int i = 0; i< structure.size(); ++i)
    {
        if ( structure[i].second.empty() )    
        {
            structure.erase(structure.begin() + i--);
        }
    }
\end{lstlisting}

\subsection{Analiza złożoności}
Kluczową rolę w kodzie odgrywają 2 funkcje: \textit{empty} oraz \textit{erase}. Funkcja \textit{empty} ma złożoność obliczeniową stałą dla jednego elementu, jednak zaimplementowana jak w powyższy sposób, wykona się dla $n$ elementów, zatem jej złożoność w tym przypadku powinna być liniowa $O(n)$. Funkcja \textit{erase} ma oczekiwaną złożoność obliczeniową także liniową $O(n)$. Przeprowadzone zostały testy dla różnych danych w pliku i zmierzone zostały czasy działania obydwu funkcji. Wyniki przedstawia poniższa tabela.

\begin{table}[H]
    \centering
    \caption{Czas działania funkcji \textit{empty} i \textit{erase} dla różnej liczby elementów}
    \label{tab: przeszukiwanie_empty_erase}
    \begin{tabular}{|c|c|c|}
    \hline
    liczba elementów & czas działania empty  [ns]  & czas działania erase [ns]   \\ \hline
    0                & 381      & 734      \\ \hline
    50000            & 1092756  & 1280059  \\ \hline
    100000           & 1486812  & 2285586  \\ \hline
    150000           & 910546   & 3468263  \\ \hline
    200000           & 1502636  & 4828635  \\ \hline
    250000           & 3267864  & 6575286  \\ \hline
    300000           & 7089495  & 7404888  \\ \hline
    350000           & 7015680  & 9116657  \\ \hline
    400000           & 2745055  & 10978655 \\ \hline
    450000           & 8849012  & 17499953 \\ \hline
    500000           & 6976789  & 27862306 \\ \hline
    550000           & 9457213  & 25233448 \\ \hline
    600000           & 7641937  & 25103366 \\ \hline
    650000           & 11925197 & 30027679 \\ \hline
    700000           & 8479812  & 28277875 \\ \hline
    750000           & 13310083 & 28351194 \\ \hline
    800000           & 14939825 & 31133886 \\ \hline
    850000           & 14277732 & 32570166 \\ \hline
    900000           & 16023509 & 34344044 \\ \hline
    950000           & 17258645 & 35836178 \\ \hline
    1000000          & 19971461 & 38559105 \\ \hline
    \end{tabular}
\end{table}



Na podstawie tabeli \ref{tab: przeszukiwanie_empty_erase} wygenerowane zostały charakterystyki działania obydwu funkcji dla różnej liczby danych w pliku. Jak pokazują poniższe wykresy, obydwie funkcje przypominają oczekiwaną charakterystykę liniową. Zatem, funkcje \textit{empty} oraz \textit{erase} w przedstawionej implementacji, mają liniowe złożoności obliczeniowe $O(n)$. 

\begin{figure}[H]
    \centering
    \subfloat[empty]
    {
        \begin{tikzpicture}[scale=0.9]
            \begin{axis}
                [
                    /pgf/number format/.cd,use comma,1000 sep={},   %europejski format
                    ylabel={czas},
                    xlabel={liczba elementów},
                    xmin=0, xmax=1000000,
                    ymin=0, ymax=19971461,
                    ymajorgrids=true, xmajorgrids=true, grid style=dashed
                ]
            
            \addplot[color=blue] table [x=x, y={create col/linear regression={y=y}}] {empty.txt};

            \addlegendentry{$ y =
            \pgfmathprintnumber{\pgfplotstableregressiona}
            \cdot x
            \pgfmathprintnumber[print sign]{\pgfplotstableregressionb}$ }

            \addplot[color=WildStrawberry, only marks, mark size=0.04cm] table {empty.txt};
            \end{axis}
        \end{tikzpicture}
    } \quad
    \subfloat[erase]
    {
        \begin{tikzpicture}[scale=0.9]
            \begin{axis}
                [
                    /pgf/number format/.cd,use comma,1000 sep={},   %europejski format
                    ylabel={czas},
                    xlabel={liczba elementów},
                    xmin=0, xmax=1000000,
                    ymin=0, ymax=38559105,
                    ymajorgrids=true, xmajorgrids=true, grid style=dashed
                ]
            
            \addplot[color=blue] table [x=x, y={create col/linear regression={y=y}}] {erase.txt};

            \addlegendentry{$ y =
            \pgfmathprintnumber{\pgfplotstableregressiona}
            \cdot x
            \pgfmathprintnumber[print sign]{\pgfplotstableregressionb}$ }

            \addplot[color=WildStrawberry, only marks, mark size=0.04cm] table {erase.txt};
            \end{axis}
        \end{tikzpicture}
    }
    \caption{Złożoności obliczeniowe funkcji \textit{empty} i  \textit{erase} w przedstawionej implementacji}
    \label{fig: empty_erase}
\end{figure}


Złożoności obydwu funkcji są liniowe, zatem całość - przeszukanie i przefiltrowanie danych powinno mieć złożoność kwadratową: $n \cdot n=n^2$. W poniższej tabeli przedstawiono zebrane pomiary działania całego algorytmu. 

\begin{table}[H]
    \centering
    \caption{Pomiary czasu działania całego algorytmu przeszukiwania i usuwania wybranych pól dla różnej liczby danych}
    \begin{tabular}{|c|c|}
    \hline
    liczba elementów & czas [ns] \\ \hline
    10000            & 99010         \\ \hline
    50000            & 2233885       \\ \hline
    100000           & 1916298094    \\ \hline
    150000           & 7481184919    \\ \hline
    200000           & 16730176118   \\ \hline
    300000           & 41073112025   \\ \hline
    400000           & 71606117902   \\ \hline
    500000           & 108592553455  \\ \hline
    600000           & 153867195664  \\ \hline
    700000           & 204008003164  \\ \hline
    800000           & 261449177263  \\ \hline
    900000           & 325313418247  \\ \hline
    1000000          & 397092989768  \\ \hline
    1010294          & 426561982926  \\ \hline
    \end{tabular}
    \label{tab: zad1}
\end{table}


\begin{figure}[H]
    \centering
    \begin{tikzpicture}[scale=1.2]
        \begin{axis}
            [
                /pgf/number format/.cd,use comma,1000 sep={},   %europejski format
                ylabel={czas},
                xlabel={liczba elementów},
                xmin=0, xmax=1000000,
                ymin=0, ymax=426561982926,
                ymajorgrids=true, xmajorgrids=true, grid style=dashed,
                legend style = {font=\tiny}
            ]
        
        % \addplot[color=blue] table [x=x, y={create col/linear regression={y=y}}] {zad1.txt};

        % \addlegendentry{$ y =
        % \pgfmathprintnumber{\pgfplotstableregressiona}
        % \cdot x
        % \pgfmathprintnumber[print sign]{\pgfplotstableregressionb}$ }

        \addplot[color=WildStrawberry, only marks, mark size=0.04cm] table {zad1.txt};
        \addplot [color=blue, domain=0:1000000] {0.378*x^2+30041.92*x-2844859911.14};
        \legend {\\ $f(x) = 0,378x^2+30041,92x-2844859911,14x$ \\ };
        \end{axis}
    \end{tikzpicture}
    \caption{Złożoność obliczeniowa całego algorytmu przeszukiwania i usuwania wybranych elementów}
    \label{fig: zad1}
\end{figure}






Jak widać na powyższym wykresie, przeszukanie i przefiltrowanie danych ma kwadratową złożoność obliczeniową $O(n^2)$. Dla ponad miliona danych, nie jest to optymalna złożoność. Łączny czas wykonywania przeszukiwania i usuwania wybranych pól zajęła: \textbf{426561982926 ns}, czyli około \textbf{7,10 min}.






\section{Analiza złożoności algorytmów sortowań}
\subsection{Przebieg eksperymentów}
Sortowane były jedynie oceny filmów. Sortowania odbyły się dla 10 000, 100 000, 500 000 oraz maksymalnej ilości danych z pliku po przefiltrowaniu. Dla każdego zestawu danych wykonano po 100 pomiarów.



\subsection{Sortowanie przez scalanie}

\begin{figure}[H]
    \centering
    \subfloat[10 000 elementów]
    {
        \begin{tikzpicture}[scale=0.9]
            \begin{axis}
                [
                    /pgf/number format/.cd,use comma,1000 sep={},   %europejski format
                    ylabel={czas},
                    xlabel={liczba elementów},
                    xmin=0, xmax=10000,
                    ymin=0, ymax=1295207,
                    ymajorgrids=true, xmajorgrids=true, grid style=dashed
                ]
            \addplot[color=WildStrawberry, only marks, mark size=0.04cm] table {przefiltrowane_dane_10tys.txt};
            \end{axis}
        \end{tikzpicture}
    } 
    \quad
    \subfloat[100 000 elementów]
    {
        \begin{tikzpicture}[scale=0.9]
            \begin{axis}
                [
                    /pgf/number format/.cd,use comma,1000 sep={},   %europejski format
                    ylabel={czas},
                    xlabel={liczba elementów},
                    xmin=0, xmax=100000,
                    ymin=0, ymax=13966928,
                    ymajorgrids=true, xmajorgrids=true, grid style=dashed
                ]
            \addplot[color=WildStrawberry, only marks, mark size=0.04cm] table {przefiltrowane_dane_100tys.txt};
            \end{axis}
        \end{tikzpicture}
    } \quad
    \subfloat[500 000 elementów]
    {
        \begin{tikzpicture}[scale=0.9]
            \begin{axis}
                [
                    /pgf/number format/.cd,use comma,1000 sep={},   %europejski format
                    ylabel={czas},
                    xlabel={liczba elementów},
                    xmin=0, xmax=500000,
                    ymin=0, ymax=70052081,
                    ymajorgrids=true, xmajorgrids=true, grid style=dashed
                ]
            \addplot[color=WildStrawberry, only marks, mark size=0.04cm] table {przefiltrowane_dane_500tys.txt};
            \end{axis}
        \end{tikzpicture}
    } \quad
    \subfloat[maksymalna liczba elementów]
    {
        \begin{tikzpicture}[scale=0.9]
            \begin{axis}
                [
                    /pgf/number format/.cd,use comma,1000 sep={},   %europejski format
                    ylabel={czas},
                    xlabel={liczba elementów},
                    xmin=0, xmax=1000000,
                    ymin=0, ymax=138481693,
                    ymajorgrids=true, xmajorgrids=true, grid style=dashed
                ]
            \addplot[color=WildStrawberry, only marks, mark size=0.04cm] table {przefiltrowane_dane_max.txt};
            \end{axis}
        \end{tikzpicture}
    }
    \caption{Sortowanie przez scalanie dla różnej liczby elementów}
    \label{fig: scalanie}
\end{figure}



\begin{table}[H]
    \centering
    \caption{Dokładny i przybliżony czas sortowania przez scalanie}
    \label{tab: czas_scalanie}
    \begin{tabular}{c|cccc|}
    \cline{2-5}
                                                             & \multicolumn{4}{c|}{liczba elementów}                                                                     \\ \cline{2-5} 
                                                             & \multicolumn{1}{c|}{10 000}  & \multicolumn{1}{c|}{100 000}  & \multicolumn{1}{c|}{500 000}  & maksymalna \\ \hline
    \multicolumn{1}{|c|}{dokładny czas sortowań {[}ns{]}}    & \multicolumn{1}{c|}{1295207} & \multicolumn{1}{c|}{13966928} & \multicolumn{1}{c|}{70052081} & 138481693  \\ \hline
    \multicolumn{1}{|c|}{przybliżony czas sortowań {[}ms{]}} & \multicolumn{1}{c|}{1,29}    & \multicolumn{1}{c|}{13,97}    & \multicolumn{1}{c|}{70,05}    & 138,48     \\ \hline
    \end{tabular}
    \end{table}


Algorytm sortowania przez scalanie wykazał złożoność obliczeniową liniową dla każdego zestawu danych. Czas sortowania dla maksymalnej liczby elementów z pliku wyniósł \textbf{138,48 ms $\approx$ 0,14 s}.





\subsection{Sortowanie szybkie}


\begin{figure}[H]
    \centering
    \subfloat[10 000 elementów]
    {
        \begin{tikzpicture}[scale=0.9]
            \begin{axis}
                [
                    /pgf/number format/.cd,use comma,1000 sep={},   %europejski format
                    ylabel={czas},
                    xlabel={liczba elementów},
                    xmin=0, xmax=10000,
                    ymin=0, ymax=1489031,
                    ymajorgrids=true, xmajorgrids=true, grid style=dashed
                ]
            \addplot[color=WildStrawberry, only marks, mark size=0.04cm] table {quick_przefiltrowane_dane_10tys.txt};
            \end{axis}
        \end{tikzpicture}
    } 
    \quad
    \subfloat[100 000 elementów]
    {
        \begin{tikzpicture}[scale=0.9]
            \begin{axis}
                [
                    /pgf/number format/.cd,use comma,1000 sep={},   %europejski format
                    ylabel={czas},
                    xlabel={liczba elementów},
                    xmin=0, xmax=100000,
                    ymin=0, ymax=11705054,
                    ymajorgrids=true, xmajorgrids=true, grid style=dashed
                ]
            \addplot[color=WildStrawberry, only marks, mark size=0.04cm] table {quick_przefiltrowane_dane_100tys.txt};
            \end{axis}
        \end{tikzpicture}
    } \quad
    \subfloat[500 000 elementów]
    {
        \begin{tikzpicture}[scale=0.9]
            \begin{axis}
                [
                    /pgf/number format/.cd,use comma,1000 sep={},   %europejski format
                    ylabel={czas},
                    xlabel={liczba elementów},
                    xmin=0, xmax=500000,
                    ymin=0, ymax=57671388,
                    ymajorgrids=true, xmajorgrids=true, grid style=dashed
                ]
            \addplot[color=WildStrawberry, only marks, mark size=0.04cm] table {quick_przefiltrowane_dane_500tys.txt};
            \end{axis}
        \end{tikzpicture}
    } \quad
    \subfloat[maksymalna liczba elementów]
    {
        \begin{tikzpicture}[scale=0.9]
            \begin{axis}
                [
                    /pgf/number format/.cd,use comma,1000 sep={},   %europejski format
                    ylabel={czas},
                    xlabel={liczba elementów},
                    xmin=0, xmax=1000000,
                    ymin=0, ymax=106918054,
                    ymajorgrids=true, xmajorgrids=true, grid style=dashed
                ]
            \addplot[color=WildStrawberry, only marks, mark size=0.04cm] table {quick_przefiltrowane_dane_max.txt};
            \end{axis}
        \end{tikzpicture}
    }
    \caption{Sortowanie przez scalanie dla różnej liczby elementów}
    \label{fig: scalanie}
\end{figure}


\begin{table}[H]
    \caption{Dokładny i przybliżony czas sortowania szybkiego}
    \label{tab: czas_szybkie}
    \begin{tabular}{c|cccc|}
    \cline{2-5}
                                                             & \multicolumn{4}{c|}{liczba elementów}                                                                     \\ \cline{2-5} 
                                                             & \multicolumn{1}{c|}{10 000}  & \multicolumn{1}{c|}{100 000}  & \multicolumn{1}{c|}{500 000}  & maksymalna \\ \hline
    \multicolumn{1}{|c|}{dokładny czas sortowań {[}ns{]}}    & \multicolumn{1}{c|}{1489031} & \multicolumn{1}{c|}{11705054} & \multicolumn{1}{c|}{57671388} & 106918054  \\ \hline
    \multicolumn{1}{|c|}{przybliżony czas sortowań {[}ms{]}} & \multicolumn{1}{c|}{1,49}    & \multicolumn{1}{c|}{11,71}    & \multicolumn{1}{c|}{57,67}    & 106,92     \\ \hline
    \end{tabular}
    \end{table}



Dla małej ilości danych (10 000), algorytm sortowania szybkiego wykazał cechy złożoności kwadratowej, natomiast dla większych ilości danych - liniowej. Czas sortowania dla maksymalnej liczby elementów z pliku wyniósł \textbf{106,92 ms $\approx$ 0,11 s}.




\subsection{Sortowanie introspektywne}

\begin{figure}[H]
    \centering
    \subfloat[10 000 elementów]
    {
        \begin{tikzpicture}[scale=0.9]
            \begin{axis}
                [
                    /pgf/number format/.cd,use comma,1000 sep={},   %europejski format
                    ylabel={czas},
                    xlabel={liczba elementów},
                    xmin=0, xmax=10000,
                    ymin=0, ymax=2873127,
                    ymajorgrids=true, xmajorgrids=true, grid style=dashed
                ]
            \addplot[color=WildStrawberry, only marks, mark size=0.04cm] table {intro_przefiltrowane_dane_10tys.txt};
            \end{axis}
        \end{tikzpicture}
    } 
    \quad
    \subfloat[100 000 elementów]
    {
        \begin{tikzpicture}[scale=0.9]
            \begin{axis}
                [
                    /pgf/number format/.cd,use comma,1000 sep={},   %europejski format
                    ylabel={czas},
                    xlabel={liczba elementów},
                    xmin=0, xmax=100000,
                    ymin=0, ymax=36543660,
                    ymajorgrids=true, xmajorgrids=true, grid style=dashed
                ]
            \addplot[color=WildStrawberry, only marks, mark size=0.04cm] table {intro_przefiltrowane_dane_100tys.txt};
            \end{axis}
        \end{tikzpicture}
    } \quad
    \subfloat[500 000 elementów]
    {
        \begin{tikzpicture}[scale=0.9]
            \begin{axis}
                [
                    /pgf/number format/.cd,use comma,1000 sep={},   %europejski format
                    ylabel={czas},
                    xlabel={liczba elementów},
                    xmin=0, xmax=500000,
                    ymin=0, ymax=148902464,
                    ymajorgrids=true, xmajorgrids=true, grid style=dashed
                ]
            \addplot[color=WildStrawberry, only marks, mark size=0.04cm] table {intro_przefiltrowane_dane_500tys.txt};
            \end{axis}
        \end{tikzpicture}
    } \quad
    \subfloat[maksymalna liczba elementów]
    {
        \begin{tikzpicture}[scale=0.9]
            \begin{axis}
                [
                    /pgf/number format/.cd,use comma,1000 sep={},   %europejski format
                    ylabel={czas},
                    xlabel={liczba elementów},
                    xmin=0, xmax=1000000,
                    ymin=0, ymax=292090659,
                    ymajorgrids=true, xmajorgrids=true, grid style=dashed
                ]
            \addplot[color=WildStrawberry, only marks, mark size=0.04cm] table {intro_przefiltrowane_dane_max.txt};
            \end{axis}
        \end{tikzpicture}
    }
    \caption{Sortowanie introspektywne dla różnej liczby elementów}
    \label{fig: introspektywne}
\end{figure}

\begin{table}[H]
    \caption{Dokładny i przybliżony czas sortowania introspektywnego}
    \label{tab: czas_introspektywne}
    \begin{tabular}{c|cccc|}
    \cline{2-5}
                                                             & \multicolumn{4}{c|}{liczba elementów}                                                                     \\ \cline{2-5} 
                                                             & \multicolumn{1}{c|}{10 000} & \multicolumn{1}{c|}{100 000}  & \multicolumn{1}{c|}{500 000}   & maksymalna \\ \hline
    \multicolumn{1}{|c|}{dokładny czas sortowań {[}ns{]}}    & \multicolumn{1}{c|}{287312} & \multicolumn{1}{c|}{36543660} & \multicolumn{1}{c|}{148902464} & 292090659  \\ \hline
    \multicolumn{1}{|c|}{przybliżony czas sortowań {[}ms{]}} & \multicolumn{1}{c|}{0,29}   & \multicolumn{1}{c|}{36,54}    & \multicolumn{1}{c|}{148,90}    & 292,09     \\ \hline
    \end{tabular}
\end{table}

Algorytm sortowania introspektywnego wykazał złożoność obliczeniową liniową dla każdego zestawu
danych. Czas sortowania dla maksymalnej liczby elementów z pliku wyniósł \textbf{292,09 ms $\approx$ 0,29 s}.


\section{Średnia wartość i mediana}
Ponadto, dla każdego zestawu danych zostały wyznaczone średnie wartości oraz mediany rankingu, których wartości zostały przedstawione w poniższej tabeli:


\begin{table}[H]
    \centering
    \caption{Średnia wartość oraz mediana wyznaczona dla każdego zestawu danych}
    \label{tab: mediana_srednia}
    \begin{tabular}{c|cccc|}
    \cline{2-5}
                                          & \multicolumn{4}{c|}{liczba elementów}                                                                  \\ \cline{2-5} 
                                          & \multicolumn{1}{c|}{10 000} & \multicolumn{1}{c|}{100 000} & \multicolumn{1}{c|}{500 000} & maksymalna \\ \hline
    \multicolumn{1}{|c|}{średnia wartość} & \multicolumn{1}{c|}{5,46}   & \multicolumn{1}{c|}{6,09}    & \multicolumn{1}{c|}{6,67}    & 6,64       \\ \hline
    \multicolumn{1}{|c|}{mediana}         & \multicolumn{1}{c|}{5}      & \multicolumn{1}{c|}{7}       & \multicolumn{1}{c|}{7}       & 7          \\ \hline
    \end{tabular}
    \end{table}


\section{Podsumowanie i wnioski}


\colorbox{WildStrawberry}{pkt 3 - opis + zaznajomienie się na jutro ; wnioski ; ewentualnie komentarze }

\section{Bibliografia}

\end{document}
