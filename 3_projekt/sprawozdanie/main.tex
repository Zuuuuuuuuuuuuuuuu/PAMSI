\documentclass[12pt]{article}
\usepackage[utf8]{inputenc}
\usepackage[T1]{fontenc}
\usepackage[polish]{babel}
\usepackage{geometry}
\usepackage{tabularx}
\usepackage[table,xcdraw,dvipsnames]{xcolor}
\usepackage{color}
\usepackage{subfig}
\usepackage{sidecap}
\usepackage{wrapfig}
\usepackage{float}
\usepackage{enumerate}
\usepackage{graphicx}
\usepackage{multirow}
\setlength{\parindent}{0pt}
\usepackage{hyperref}
\usepackage{titlesec}
\titlelabel{\thetitle.\quad}
\usepackage{amsmath}
\usepackage{anyfontsize}
\usepackage{indentfirst}
\usepackage{listings}
\usepackage{multicol}
\usepackage{pgfplots}
\usepackage{listings}
\usepackage{fancyhdr}
\usepackage{pgfplotstable}
\newgeometry{tmargin=1.8cm,bmargin=1.8cm,lmargin =1.8cm,rmargin=1.8cm}

\newgeometry{tmargin=1.8cm,bmargin=1.8cm,lmargin =1.8cm,rmargin=1.8cm}
\pagestyle{fancy}
\fancyhf{}
\rhead{\textit{Zuzanna Mejer} }
\lhead{\textit{Kółko i krzyżyk} }
\cfoot{\thepage}


\begin{document}
\renewcommand{\figurename}{Rys.}
\renewcommand{\tablename}{Tab.}
    
\begin{titlepage}
    \begin{figure}
        \centering
        \includegraphics[width=18cm]{logo-PWr.png}
        
        \label{fig:pwr}
    \end{figure}
        \begin{center}
            \huge Wydział Elektroniki, Fotoniki i Mikrosystemów \\ 
            \vspace{40pt}
            \huge Projektowanie Algorytmów i Metody Sztucznej Inteligencji  \\
        \end{center}
        \vspace{60pt}
        \hrule
        \vspace{1pt}
        \hrule
        \begin{center}
            {\fontsize{40}{50}\selectfont Kółko i krzyżyk\\ }
            \vspace{10pt}
            {\fontsize{25}{25}\selectfont Projekt 3 - zadanie na ocenę bdb  }
        \end{center}
        \hrule
        \vspace{1pt}
        \hrule
        \begin{flushright}
            \vspace{65pt}
            \textit{\Large Prowadzący:}\\
            
            \Large Dr inż. Łukasz Jeleń\\
            \vspace{10pt}
            \textit{\Large Wykonała:}\\
            \Large Zuzanna Mejer, 259382 \\
        
        \end{flushright}
        \vspace{65pt}
        \begin{center}
            \large Wrocław, \today r.
        \end{center}
    \end{titlepage}
\tableofcontents
\newpage

\section{Wprowadzenie}
Zadanie polegało na zaimplementowaniu gry Kółko i Krzyżyk z wykorzystaniem technik sztucznej inteligencji - algorytmu Minimax z alfa-beta cięciami. Użytkownik miał mieć możliwości definiowania rozmiaru planszy wraz z ilością znaków w rzędzie. 

\section{Opis tworzonej gry}
Kółko i Krzyżyk to gra o sumie zerowej, w której bierze udział dwóch użytkowników. W tym przypadku jest to gracz i komputer. Obydwoje gracze dążą do ustawienia swoich znaków w rzędzie, kolumnie lub po przekątnej tak, aby zachować ciągłość swoich znaków (żeby znak przeciwnika nie przeszkodził w tworzeniu linii). Jednocześnie, gracze dążą do tego, żeby przerwać ciągłość znaków przeciwnika w linii. W celu stworzenia gry w Kółko i Krzyżyk przyjęto parę założeń:
\begin{itemize}
    \item Grę zawsze rozpoczyna Kółko.
    \item Użytkownik nie może wybrać czy jest Kółkiem, czy Krzyżykiem. Użytkownik zawsze jest Kółkiem i zatem zawsze rozpoczyna grę.
    \item Przed rozpoczęciem gry użytkownik musi zdefiniować rozmiar kwadratowej planszy, to znaczy ile chce mieć kratek w rzędzie.
    \item W zależności od wybranego rozmiaru planszy, do wygranej należy ułożyć w rzędzie, kolumnie lub po przekątnej tyle znaków, ile wynosi liczba kratek w jednej linii na planszy. To znaczy, że trzeba ze znaków ułożyć ciągłą linię od jednego końca planszy, do drugiego. 
\end{itemize}

\section{Graficzny interfejs użytkownika}
W celu ułatwienia użytkownikowi obsługi gry, stworzono graficzny interfejs. Wykorzystano w tym celu bibliotekę programistyczną \textit{SFML} (Simple and Fast Multimedia Library).

\subsection{Kółko i Krzyżyk}
Kółko i Krzyżyk zostały zaimplementowane i rysowane jako litery ,,X'' i ,,O'' czcionką Arial o odpowiednio dobranym rozmiarze tak, żeby wpasowywały się w środek kratki, niezależnie od liczby kratek na planszy. Przedstawia to poniższy kod:

\lstset{language=C++, firstnumber=1, keywordstyle=\color{blue}, numbers=left, frame = single}
\begin{lstlisting}
    sf::Font font;
    if (!font.loadFromFile("arial.ttf"))
    {
        std::cout << "Nie mamy takiej czcionki\n";
    }

    sf::Text circle;
    circle.setFont(font);
    circle.setString("O");
    circle.setCharacterSize(one_cell_size);
    circle.setFillColor(sf::Color::Black);
    circle.setLineSpacing(0);
    circle.setLetterSpacing(0);

    
    sf::Text cross;
    cross.setFont(font);
    cross.setString("X");
    cross.setCharacterSize(one_cell_size);
    cross.setFillColor(sf::Color::Black);
    cross.setLineSpacing(0);
    cross.setLetterSpacing(0);

\end{lstlisting}

\subsection{Rysowanie planszy}
Rysowanie planszy odbywa się tylko raz na początku gry. Po podaniu przez użytkownika rozmiaru planszy, program dostosowuje rozmiar jednej kratki do wielkości okna, zdefiniowanej jako zmienna globalna: \textit{constexpr int one\_side\_number\_of\_cells = 800;}. Następnie rysuje kratki za pomocą prostokątów o odpowiednim rozmiarze i odpowiedniej rotacji. Jest to przedstawione poniżej:


\lstset{language=C++, firstnumber=1, keywordstyle=\color{blue}, numbers=left, frame = single}
\begin{lstlisting}
void draw_board(sf::RenderWindow &window, int cells)
{
    int one_cell_size = window.getSize().x / cells;
    int tmp = one_cell_size;
    sf::RectangleShape rectangle(sf::Vector2f(5, one_side_number_of_cells));
    rectangle.setFillColor(sf::Color::Black);

    for (int i = 1; i <= cells - 1; i++)
    {
        rectangle.setPosition(one_cell_size, 0);
        window.draw(rectangle);
        one_cell_size = one_cell_size + tmp;
    }
    
    one_cell_size = window.getSize().x / cells;
    rectangle.rotate(-90);
    for (int i = 1; i <= cells - 1; i++)
    {
        rectangle.setPosition(0, one_cell_size);
        window.draw(rectangle);
        one_cell_size = one_cell_size + tmp;
    }
}
\end{lstlisting}





\subsection{Obsługa ,,wydarzeń''}

Każdorazowe kliknięcie myszką na planszę jest obsługiwane przez \textit{sf::Event}. Po wykryciu kliknięcia, obliczana jest pozycja myszki i rysowany jest znak kółka w odpowiednim miejscu. Niestety program nie jest odporny na błędy w postaci kilkukrotnego klikania myszką w jedno miejsce. W takim przypadku na planszy pojawi się więcej krzyżyków tak, jakby gra toczyła się dalej. Kod przedstawiony poniżej opisuje także obsługę zamknięcia okna z grą, co jest jednoznaczne z zakończeniem gry.

\lstset{language=C++, firstnumber=1, keywordstyle=\color{blue}, numbers=left, frame = single}
\begin{lstlisting}
while (board.pollEvent(event))
{
    if (event.type == sf::Event::Closed)
        board.close();

    if (event.type == sf::Event::MouseButtonPressed)
    {
        if (event.mouseButton.button == sf::Mouse::Left)
        {
            sf::Vector2i position = sf::Mouse::getPosition(board);
            int column = position.x / one_cell_size;
            int row = position.y / one_cell_size;
            tab[row][column] = 'O';
            circle.setPosition(column * one_cell_size + 
            (circle.getCharacterSize() * 0.1), 
            row * one_cell_size + (circle.getCharacterSize() * -0.13));       
            board.draw(circle);
            board.display();
        }
    }
}

\end{lstlisting}

\section{Techniki AI}
\subsection{Wstęp}
Ogólnie, wykorzystane techniki można przedstawić za pomocą grafów, gdzie każde kolejne rozgałęzienie pokazuje możliwe ruchy gracza i komputera. Aby wydedukować jaki ruch będzie najoptymalniejszy do wykonania przez komputer, musi on z zadanej pozycji obliczyć wszystkie możliwości i wybrać tę, która da mu najszybsze zwycięstwo. W tym celu wykorzystano strategię Minimax, która ma na celu maksymalizację szans na wygraną jednego użytkownika (w tym przypadku komputera) i minimalizację szans na wygraną przeciwnika (w tym przypadku gracza). Ze względu na to, że pełen algorytm Minimax, czyli rozważanie wszystkich przypadków i rozwijanie drzewa do końca byłby zbyt kosztowny i długotrwały, zastosowano funkcję heurystyczną oraz alfa-beta cięcia. Funkcja heurystyczna ma na celu jedynie oszacowanie najoptymalniejszego ruchu, nie gwarantuje przy tym, że będzie to ruch prawidłowy czy najlepszy. Natomiast alfa-beta cięcia pozwalają wyeliminować z analizy węzły nie mające wpływu na wartośc przodków. To znaczy, że jeżeli w jakiejś części grafu znaleziono jakąś lepszą wartość i jakąś gorszą wartość, to ten węzeł, który wskazuje gorszą wartość, zostanie porzucony i komputer nie będzie analizować dalej tej gałęzi. Techniki AI zostały zaimplementowane jako 3 funkcje: \textit{rate\_positions} - do liczenia i nadawania wag poszczególnym rozwiązaniom, \textit{minimax\_alpha\_beta} - do wybierania najlepszej wagi oraz \textit{best\_ai\_move} do wywoływania algorytmu Minimax i do zwracania najlepszego ruchu dla komputera. Funkcje te zostały przedstawione i opisane poniżej.

\subsection{Funkcja \textit{rate\_positions} - do liczenia i nadawania wag poszczególnym rozwiązaniom}
\lstset{language=C++, firstnumber=1, keywordstyle=\color{blue}, numbers=left, frame = single}
\begin{lstlisting}
int rate_positions(int &number_of_cells, char **tab)       
{
    int position_rating = 0;
    int tmp = 0;

    for (int i=0; i < number_of_cells; i++)                
    {
        tmp = 0;
        for (int j=0; j < number_of_cells; j++)
        {
            if (tab[i][j] == 'X')
                tmp++;
            
            else if (tab[i][j] == 'O')
            {
                tmp = 0;
                break;
            } 
        }

        if (tmp)
            position_rating += pow(10, tmp);    

    
    for (int i=0; i < number_of_cells; i++)                
    {
        tmp = 0;
        for (int j=0; j < number_of_cells; j++)
        {
            if (tab[j][i] == 'X')
                tmp++;
            
            else if (tab[j][i] == 'O')
            {
                tmp = 0;
                break;
            } 
        }
        
        if (tmp)
            position_rating += pow(10, tmp);   
    }

    tmp = 0;
    for (int j=0; j < number_of_cells; j++)         
    {
        if (tab[j][j] == 'X')
            tmp++;
        
        else if (tab[j][j] == 'O')
        {
            tmp = 0;
            break;
        } 
    
        if (tmp)
            position_rating += pow(10, tmp);    
    }
    

    for (int i = 0, j = number_of_cells - 1; i < number_of_cells; i++, j--)     
    {
        tmp = 0;
        if (tab[i][j] == 'X')
            tmp++;

        else if (tab[i][j] == 'O')
        {
            tmp = 0;
            break;
        }

        if (tmp)
            position_rating += pow(10, tmp);
    }
    return position_rating;
    }
}
\end{lstlisting}

Funkcja \textit{rate\_positions} jest wywoływana w funkcji \textit{minimax\_alpha\_beta} w dwóch przypadkach: po zakończeniu gry albo gdy głębokość osiągnie wartość 0, czyli przy ostatnim wywołaniu Minimaxa. W obydwu przypadkach komputer analizuje sytuację na planszy, kiedy wszystkie pola są zajęte. Zatem funkcja ta składa się z przeszukiwania kolejno rzędów, kolumn i dwóch przekątnych. Jeżeli w którejś z nich znajdzie ciągłość znaków ,,X'' (czyli tych, którymi gra komputer), to nadaje takiemu rozwiązaniu odpowiednio dużą wagę. Jeżeli natomiast znajdzie rozwiązania, w których na przykład znaki ,,X'' są rozdzielone znakiem ,,O'' lub też rozwiązania, w których znaki ,,O'' zachowują ciągłość w linii, to nie nadaje im wag.  

\subsection{Funkcja \textit{minimax\_alpha\_beta} - do wybierania najlepszej wagi}

\lstset{language=C++, firstnumber=1, keywordstyle=\color{blue}, numbers=left, frame = single}
\begin{lstlisting}
int minimax_alpha_beta(char current_player, int depth, int a, int b,
 int &number_of_cells, char **tab)
{
    check_win(number_of_cells, tab);

    if (winner != '-')                
    {
        if (current_player == 'X')      
            return pow(10,8);            // max wartosc
        else
            return -pow(10,8);       // min wartosc
    }

    if (is_finished(number_of_cells, tab) || depth == 0)
    {
        if (current_player == 'X')
            return rate_positions(number_of_cells, tab);    
        else
            return -rate_positions(number_of_cells, tab);
    }

    int best_score;

    if (current_player == 'X')
    {
        current_player = 'O';          // zamiana graczy
        best_score = infinity;
    }
    else
    {
        current_player = 'X';
        best_score = -infinity;
    }

    int tmp;
    for (int i = 0; i < number_of_cells; i++)
    {
        for (int j = 0; j < number_of_cells; j++)
        {
            if (tab[i][j] == '-')
            {
                if (current_player == 'X')
                {
                    tab[i][j] = 'X';
                    tmp = minimax_alpha_beta(current_player, depth-1, a, 
                    b, number_of_cells, tab);
                    if (best_score < tmp)
                        best_score = tmp;
                    
                    if (a < best_score)
                        a = best_score;

                    tab[i][j] = '-';
                    if (a >= b)
                        return best_score;
                }
                else
                {
                    tab[i][j] = 'O';
                    tmp = minimax_alpha_beta(current_player, depth-1, a, 
                    b, number_of_cells, tab);
                    if (best_score > tmp)
                        best_score = tmp;
                    
                    if (a > best_score)
                        a = best_score;

                    tab[i][j] = '-';
                    if (a >= b)
                        return best_score;
                }
            }
        }
    }
    return best_score;
}
\end{lstlisting}

Funkcja \textit{minimax\_alpha\_beta} najpierw sprawdza czy funkcja już rozwinęła ścieżkę, w której doszło do zwycięstwa. Jeżeli tak, rozwiązaniom, w których zwyciężył ,,X'' (komputer) przypisuje maksymalną wartość, a rozwiązaniom, w których zwyciężyło ,,O'' (użytkownik), przypisuje minimalną wartość. Następnie, jeżeli gra jest skończona lub algorytm doszedł już do wskazanej głębokości, oceniana jest pozycja. Potem w każdorazowym wywołaniu funkcji, następuje zamiana gracza, którego ruch jest rozważany. Następnie, algorytm przechodzi po komórkach tablicy i w wolne miejsca wpisuje odpowiedni znak po to, by rekurencyjnie znów się wywołać, z głębokością pomniejszoną o 1. W ten sposób algorytm dochodzi do końcowej głębokości (do pewnego rozwinięcia drzewa), z czego każda rozważana możliwość zagrania ma swoją wartość. Na koniec algorytm stosuje alfa-beta cięcia, to znaczy, że będąc na końcowej głębokości, wraca do rodzica, wybierając najlepszą ścieżkę i odrzucając inne rozwiązania.


\subsection{Funkcja {best\_ai\_move} - do wywoływania algorytmu Minimax i do zwracania pozycji najlepszego ruchu dla komputera}


\lstset{language=C++, firstnumber=1, keywordstyle=\color{blue}, numbers=left, frame = single}
\begin{lstlisting}
std::pair<int, int> best_ai_move(int depth, int &number_of_cells, char **tab)
{
    int best_score = minus_infinity;
    int tmp;
    int set_i;
    int set_j;

    for (int i = 0; i < number_of_cells; i++)
    {
        for (int j = 0; j < number_of_cells; j++)
        {
            if (tab[i][j] == '-')
            {
                tab[i][j] = 'X';
                tmp = minimax_alpha_beta('X', depth, -infinity,
                infinity, number_of_cells, tab);
                tab[i][j] = '-';
                if (tmp > best_score)
                {
                    best_score = tmp;
                    set_i = i;
                    set_j = j;
                }
            }
        }
    }

    if (set_i < number_of_cells && set_j < number_of_cells)
        tab[set_i][set_j] = 'X';

    return std::make_pair(set_i, set_j);
}

\end{lstlisting}

Funkcja ta przechodzi kolejno po komórkach planszy i gdy napotka puste pole, wpisuje tam znak ,,X'', po czym wywołuje funkcje \textit{minimax\_alpha\_beta}, która rozbudowuje drzewo i wybiera najlepszą wagę, którą zapisuje do zmiennej \textit{tmp}. Po cofnięciu wprowadzonej zmiany do komórki planszy, funkcja ta porównuje zmienną \textit{tmp} z inną ustaloną wartością (minimalną). Jeżeli znajdzie pozycję większą od ustalonej, wpisuje i zwraca w którą komórkę komputer wykona ruch.


\section{Podsumowanie i wnioski}
\begin{enumerate}
    \item SFML jest bogato wyposażoną biblioteką, którą wykorzystuje się do tworzenia na przykład graficznych interfejsów czy gier. Nie oferuje jednak gotowych rozwiązań do tworzenia guzików, które byłyby przydatne do komunikacji z użytkownikiem. Zamiast tego, w projekcie wykorzystano w tym celu terminal.
    \item W programie znajduje się niedopatrzenie, które umożliwia użytkownikowi wpisanie znaku nieskończenie wiele razy w tę samą komórkę planszy. 
    \item Program działa poprawnie. Dla każdego testowanego rozmiaru planszy zaimplementowane techniki AI blokują wygraną użytkownika i dążą do wygranej komputera.
    \item Przed rozpoczęciem należy ustawić poziom trudności gry. Jest to jednoznaczne z ustawieniem głębokości, do jakiej komputer będzie liczyć różne warianty. Optymalny wybór poziomu trudności zależy od rozmiaru planszy. Dla planszy 3x3 i głębokości 5, ruch komputera następuje praktycznie od razu, natomiast dla planszy 5x5 i tej samej głębokości, na ruch komputera trzeba poczekać około 50 sekund.
    \item Jest jeszcze wiele rzeczy do dopracowania bądź zaimplementowania w omawianym projekcie, na przykład: dodanie możliwości wyboru znaku przez użytkownika, lepsze oznaczenie wygranej z wykorzystaniem graficznego interfejsu, czy na przykład możliwość definiowania liczby tych samych znaków w linii potrzebnej do wygranej.
\end{enumerate}

\section{Literatura}
\begin{itemize}
    \item https://www.sfml-dev.org/ (dostęp: 8.06.2022)
    \item https://en.wikipedia.org/wiki/Minimax (dostęp: 8.06.2022)
    \item Notatki z wykładów
\end{itemize}

\end{document}