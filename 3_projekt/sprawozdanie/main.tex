\documentclass[12pt]{article}
\usepackage[utf8]{inputenc}
\usepackage[T1]{fontenc}
\usepackage[polish]{babel}
\usepackage{geometry}
\usepackage{tabularx}
\usepackage[table,xcdraw,dvipsnames]{xcolor}
\usepackage{color}
\usepackage{subfig}
\usepackage{sidecap}
\usepackage{wrapfig}
\usepackage{float}
\usepackage{enumerate}
\usepackage{graphicx}
\usepackage{multirow}
\setlength{\parindent}{0pt}
\usepackage{hyperref}
\usepackage{titlesec}
\titlelabel{\thetitle.\quad}
\usepackage{amsmath}
\usepackage{anyfontsize}
\usepackage{indentfirst}
\usepackage{listings}
\usepackage{multicol}
\usepackage{pgfplots}
\usepackage{listings}
\usepackage{fancyhdr}
\usepackage{pgfplotstable}
\newgeometry{tmargin=1.8cm,bmargin=1.8cm,lmargin =1.8cm,rmargin=1.8cm}

\newgeometry{tmargin=1.8cm,bmargin=1.8cm,lmargin =1.8cm,rmargin=1.8cm}
\pagestyle{fancy}
\fancyhf{}
\rhead{\textit{Zuzanna Mejer} }
\lhead{\textit{Kółko i krzyżyk} }
\cfoot{\thepage}


\begin{document}
\renewcommand{\figurename}{Rys.}
\renewcommand{\tablename}{Tab.}
    
\begin{titlepage}
    \begin{figure}
        \centering
        \includegraphics[width=18cm]{logo-PWr.png}
        
        \label{fig:pwr}
    \end{figure}
        \begin{center}
            \huge Wydział Elektroniki, Fotoniki i Mikrosystemów \\ 
            \vspace{40pt}
            \huge Projektowanie Algorytmów i Metody Sztucznej Inteligencji  \\
        \end{center}
        \vspace{60pt}
        \hrule
        \vspace{1pt}
        \hrule
        \begin{center}
            {\fontsize{40}{50}\selectfont Kółko i krzyżyk\\ }
            \vspace{10pt}
            {\fontsize{25}{25}\selectfont Projekt 3 - zadanie na ocenę bdb  }
        \end{center}
        \hrule
        \vspace{1pt}
        \hrule
        \begin{flushright}
            \vspace{65pt}
            \textit{\Large Prowadzący:}\\
            
            \Large Dr inż. Łukasz Jeleń\\
            \vspace{10pt}
            \textit{\Large Wykonała:}\\
            \Large Zuzanna Mejer, 259382 \\
        
        \end{flushright}
        \vspace{65pt}
        \begin{center}
            \large Wrocław, \today r.
        \end{center}
    \end{titlepage}
\tableofcontents
\newpage

\section{Wprowadzenie}
Zadanie polegało na zaimplementowaniu gry Kółko i Krzyżyk z wykorzystaniem technik sztucznej inteligencji - algorytmu Minimax z alfa-beta cięciami. Użytkownik miał mieć możliwości definiowania rozmiaru planszy wraz z ilością znaków w rzędzie. 

\section{Opis tworzonej gry}
Kółko i Krzyżyk to gra o sumie zerowej, w której bierze udział dwóch użytkowników. W tym przypadku jest to gracz i komputer. Obydwoje gracze dążą do ustawienia swoich znaków w rzędzie, kolumnie lub po przekątnej tak, aby zachować ciągłość swoich znaków (żeby znak przeciwnika nie przeszkodził w tworzeniu linii). Jednocześnie, gracze dążą do tego, żeby przerwać ciągłość znaków przeciwnika w linii. W celu stworzenia gry w Kółko i Krzyżyk przyjęto parę założeń:
\begin{itemize}
    \item Grę zawsze rozpoczyna Kółko.
    \item Użytkownik nie może wybrać czy jest Kółkiem, czy Krzyżykiem.Użytkownik zawsze jest Kółkiem i zatem zawsze rozpoczyna grę.
    \item Przed rozpoczęciem gry użytkownik musi zdefiniować rozmiar kwadratowej planszy, to znaczy ile chce mieć kratek w rzędzie.
    \item W zależności od wybranego rozmiaru planszy, do wygranej należy ułożyć w rzędzie, kolumnie lub po przekątnej tyle znaków, ile wynosi liczba kratek w jednej linii na planszy. To znaczy, że trzeba ze znaków ułożyć ciągłą linię od jednego końca planszy, do drugiego. 
\end{itemize}

\section{Graficzny interfejs użytkownika}
W celu ułatwienia użytkownikowi obsługi gry, stworzono graficzny interfejs. Wykorzystano w tym celu bibliotekę programistyczną \textit{SFML} (Simple and Fast Multimedia Library).

\subsection{Kółko i Krzyżyk}
Kółko i Krzyżyk zostały zaimplementowane i rysowane jako litery ,,X'' i ,,O'' czcionką Arial o odpowiednio dobranym rozmiarze tak, żeby wpasowywały się w środek kratki, niezależnie od liczby kratek na planszy. Przedstawia to poniższy kod:

\lstset{language=C++, firstnumber=1, keywordstyle=\color{blue}, numbers=left, frame = single}
\begin{lstlisting}
    sf::Font font;
    if (!font.loadFromFile("arial.ttf"))
    {
        std::cout << "Nie mamy takiej czcionki\n";
    }

    sf::Text circle;
    circle.setFont(font);
    circle.setString("O");
    circle.setCharacterSize(one_cell_size);
    circle.setFillColor(sf::Color::Black);
    circle.setLineSpacing(0);
    circle.setLetterSpacing(0);

    
    sf::Text cross;
    cross.setFont(font);
    cross.setString("X");
    cross.setCharacterSize(one_cell_size);
    cross.setFillColor(sf::Color::Black);
    cross.setLineSpacing(0);
    cross.setLetterSpacing(0);

\end{lstlisting}

\subsection{Rysowanie planszy}
Rysowanie planszy odbywa się tylko raz na początku gry. Po podaniu przez użytkownika rozmiaru planszy, program dostosowuje rozmiar jednej kratki do wielkości okna, zdefiniowanej jako zmienna globalna: \textit{constexpr int one\_side\_number\_of\_cells = 800;}. Następnie rysuje kratki za pomocą prostokątów o odpowiednim rozmiarze i odpowiedniej rotacji. Jest to przedstawione poniżej:


\lstset{language=C++, firstnumber=1, keywordstyle=\color{blue}, numbers=left, frame = single}
\begin{lstlisting}
void draw_board(sf::RenderWindow &window, int cells)
{
    int one_cell_size = window.getSize().x / cells;
    int tmp = one_cell_size;
    sf::RectangleShape rectangle(sf::Vector2f(5, one_side_number_of_cells));
    rectangle.setFillColor(sf::Color::Black);

    for (int i = 1; i <= cells - 1; i++)
    {
        rectangle.setPosition(one_cell_size, 0);
        window.draw(rectangle);
        one_cell_size = one_cell_size + tmp;
    }
    
    one_cell_size = window.getSize().x / cells;
    rectangle.rotate(-90);
    for (int i = 1; i <= cells - 1; i++)
    {
        rectangle.setPosition(0, one_cell_size);
        window.draw(rectangle);
        one_cell_size = one_cell_size + tmp;
    }
}
\end{lstlisting}





\subsection{Obsługa ,,wydarzeń''}

Każdorazowe kliknięcie myszką na planszę jest obsługiwane przez \textit{sf::Event}. Po wykryciu kliknięcia, obliczana jest pozycja myszki i rysowany jest znak kółka w odpowiednim miejscu. Niestety program nie jest odporny na błędy w postaci kilkukrotnego klikania myszką w jedno miejsce. W takim przypadku na planszy pojawi się więcej krzyżyków tak, jakby gra toczyła się dalej. Kod przedstawiony poniżej opisuje także obsługę zamknięcia okna z grą, co jest jednoznaczne z zakończeniem gry.

\lstset{language=C++, firstnumber=1, keywordstyle=\color{blue}, numbers=left, frame = single}
\begin{lstlisting}
while (board.pollEvent(event))
{
    if (event.type == sf::Event::Closed)
        board.close();

    if (event.type == sf::Event::MouseButtonPressed)
    {
        if (event.mouseButton.button == sf::Mouse::Left)
        {
            sf::Vector2i position = sf::Mouse::getPosition(board);
            int column = position.x / one_cell_size;
            int row = position.y / one_cell_size;
            tab[row][column] = 'O';
            circle.setPosition(column * one_cell_size + 
            (circle.getCharacterSize() * 0.1), 
            row * one_cell_size + (circle.getCharacterSize() * -0.13));       
            board.draw(circle);
            board.display();
        }
    }
}

\end{lstlisting}


\subsection{Oznaczenie wygranej}

\colorbox{WildStrawberry}{ohoho}

\section{Techniki AI}

\section{Podsumowanie i wnioski}
\begin{enumerate}
    \item o tych kliknieciach ze nie ok kilkukrotne
    \item no czy dziala To
    \item i dla jakich glebokosci
    \item ze sfml jest super, ale szkoda ze nie ma buttons
    \item gdybym robila dopracowanie to jeszcze bym dodala mozliwosci 2: 1. wybranie przez uzytkownika kolko czy krzyzyk i 2. moze ile tych znakow w rzedzie do wygranej itd.
\end{enumerate}

\section{Literatura}
\begin{itemize}
    \item https://www.sfml-dev.org/ (dostęp: 8.06.2022)
    \item https://en.wikipedia.org/wiki/Minimax (dostęp: 8.06.2022)
    \item Notatki z wykładów
\end{itemize}

\end{document}